\section{QUA-Compliance}
\label{ch:quacompliance}

To make a service compliant for use in the QUA-Kit, a service is subject to additional constraints that are described here.
A qua-view-complient service must support one or more of following working modes:
\begin{itemize}
\item 
    \texttt{scenario} -- a service gets a scenario as input and returns a single result for a whole scenario;
\item 
    \texttt{objects} -- a service gets a scenario and ids of individual objects and returns a result per every object;
\item 
    \texttt{points} -- a service gets a scenario and a grid of points, and returns a result per each point on the grid;
\item 
    \texttt{new} -- a service updates or creates a new geometry.
\end{itemize}

\subsubsection{Registering}

In the following section, we will introduce the required fields and formats for a qua-view-compliant service. A full registration message can be seen in Listing \ref{lst:quacompliance}.

\begin{lstlisting}[caption={Registering a QUA-compliant service}, label={lst:quacompliance}]
{
  run : "RemoteRegister",
  description : "I am a random service", // obligatory
  serviceName : "RandomQUA",
  callID : 1, // obligatory
  qua-view-compliant : true, // obligatory
  inputs : {
    OPT geomID : number, // obligatory for all modes except "new"
    mode : string, // obligatory
    points : attachment,
    some other input : number,
    some 4th input : number
  },
  outputs : {
    unit : string,
    mode : string,
    XOR value : number,
    XOR values : attachment,
    OPT scenario_id : long,
    OPT timestamp_accessed : int,
    OPT timestamp_modified : int
  }
  constraint : {
      some other input : {
          min : 42,
          max : 100,
          integer : true
      }
  },
  exampleCall : {
    {
      run : "RandomQUA",
      ScId : 123,
      mode : "points",
      points : {
        format : "float32 array",
        name : "points"
        attachment {
          length : 100,
          position : 1,
          checksum : "2929bead3ee3cf55113ec9aade2b6add"
        }
      }
      some other input : 42,
      some 4th input : 13.37
    }
  }
}
\end{lstlisting}

\paragraph{Metadata}
When registering a service, the attribute \texttt{description} of the RemoteRegister service becomes obligatory. Furthermore, the boolean field \texttt{qua-view-compliant} must be provided and set to \texttt{True} at root-level.

\paragraph{Constraints}
The optional field \texttt{constraints} is introduced. The field specifies the possible values for service inputs. Each input possibly represents an element in the constraints-dictionary. Depending on the input type, the constraint takes the following formats:

\begin{lstlisting}[caption={A string constraint}, label={lst:constraints}]
{
  constraint : {
      some number input : {
          min : 42,
          max : 100,
          integer : true
      },
      some string input : ["foo", "bar"]
  }
}
\end{lstlisting}


\begin{itemize}
	\item If the input field is of type \texttt{number}, then its corresponding constraint-value is a dictionary with up to three keys: \texttt{min}, \texttt{max} and \texttt{integer} specifying the minimum and maximum of the valid input range as well as whether the input must be an integer. In Listings \ref{lst:quacompliance} and \ref{lst:constraints}, the number-input \texttt{some other input} is constrained to $[42,100] \cap \mathbb{Z}$.
	\item If the input field is of type \texttt{string}, then its corresponding constraint-value is a list of possible values for this string as seen in Listing \ref{lst:constraints} where \texttt{some string input} has two valid input values: \emph{foo} and \emph{bar}
\end{itemize}

\subsubsection{Inputs}
While services originally can have arbitrary inputs, in QUA-compliant services, two input fields become obligatory: \texttt{mode} and \texttt{ScID}.The mode is how the service should operate, normally it operates on a \texttt{scenario}, \texttt{objects} or \texttt{points} level; or creates a new scenario in the mode \texttt{new}.

The scenario-id specifies the scenario which should be analyzed by the service. If the mode is either \texttt{objects} or \texttt{points}, a binary attachment must be included in the service call:
\begin{itemize}
  \item \texttt{points}: Points are serialized to a binary file of \texttt{float32} values wherein three consecutive values correspond to one three dimensional vector.
  \item \texttt{objects} are serialized to a binary file \texttt{ulong64} values wherein each value corresponds to one object id contained in the specified scenario.
\end{itemize}

Examples are given in Listings \ref{lst:quacompliantinput:scenario}, \ref{lst:quacompliantinput:objects}, \ref{lst:quacompliantinput:points}, \ref{lst:quacompliantinput:new}.

\begin{lstlisting}[caption={A qua-compliant service run request for mode \texttt{scenario}}, label={lst:quacompliantinput:scenario}]
{
  run : "RandomQUA",
  callID : 0,
  ScId : 123,
  mode : "scenario"
}
\end{lstlisting}

\begin{lstlisting}[caption={A qua-compliant service run request for mode \texttt{points}}, label={lst:quacompliantinput:points}]
{
  run : "RandomQUA",
  callID : 0,
  ScId : 123,
  mode : "points",
  points : {
    format : "float32 array",
    name : "points",
    attachment {
      length : 3072,
      position : 1,
      checksum : "2929bead3ee3cf55113ec9aade2b6add"
    }
  }
}
\end{lstlisting}

\begin{lstlisting}[caption={A qua-compliant service run request for mode \texttt{objects}}, label={lst:quacompliantinput:objects}]
{
  run : "RandomQUA",
  callID : 0,
  ScId : 123,
  mode : "objects",
  objVals : {
    format : "ulong64 array",
    name : "objects",
    attachment {
      length : 2048,
      position : 1,
      checksum : "50752c7cb358a5fffd913d9aa3605433"
    }
  }
}
\end{lstlisting}


\begin{lstlisting}[caption={A qua-compliant service run request for mode \texttt{new}}, label={lst:quacompliantinput:new}]
{
  run : "RandomQUA",
  callID : 0,
  mode : "new" // Note how the call does not contain a ScId
}
\end{lstlisting}

\subsubsection{Outputs}
The output of a qua-compliant service depends on the mode it operates in. The header is therefore appended with the field \texttt{mode} which corresponds to the given input mode. The additional fields are defined as follows:

\begin{itemize}
  \item \texttt{scenario}: The fields \texttt{unit : string} and \texttt{value : number} have to be added at root-level to the output header.

  \item \texttt{objects}: The field \texttt{unit : string} is added at root level to the output header. Furthermore an attachment named \texttt{values} is added that consists of a numeric array in which the service results for all objects are stored in input-order.

  \item \texttt{points}: The field \texttt{unit : string} is added at root level to the output header. Furthermore an attachment named \texttt{values} is added that consists of a numeric array in which the service results for all points are stored in input-order.

  \item \texttt{new}: The header is appended at root-level with the following fields:
  \begin{lstlisting}
{
  scenario_id : long, // the id of the newly created scenario
  timestamp_accessed : long, // the timestamp at which the scenario was accessed if the ScID-field was set in the input
  timestamp_modified : long // the timestamp at which the new scenario was stored
}
  \end{lstlisting}
\end{itemize}

Example outputs are given in Listings \ref{lst:quacompliantresult:scenario}, \ref{lst:quacompliantresult:objects}, \ref{lst:quacompliantresult:points}, \ref{lst:quacompliantresult:new}.

\begin{lstlisting}[caption={A qua-compliant service output for mode \texttt{scenario}}, label={lst:quacompliantresult:scenario}]
{
  result : {
    unit : "metre",
    mode : "scenario",
    value : 15.0
  },
  callID : 213,
  duration : 10.2
}
\end{lstlisting}

\begin{lstlisting}[caption={A qua-compliant service output for mode \texttt{objects}}, label={lst:quacompliantresult:objects}]
{
  result : {
    unit : "metre",
    mode : "objects"
  },
  callID : 213,
  duration : 10.2,
  values : {
    name : "values",
    format : "float32 array",
    attachment : {
      length: 1024,
      checksum: "2929bead3ee3cf55113ec9aade2b6add",
      position: 1
    }
  }
}
\end{lstlisting}

\begin{lstlisting}[caption={A qua-compliant service output for mode \texttt{points}}, label={lst:quacompliantresult:points}]
{
  result : {
    unit : "metre",
    mode : "points"
  },
  callID : 213,
  duration : 10.2,
  values : {
    name : "values",
    format : "float32 array",
    attachment : {
      length: 1024,
      checksum: "50752c7cb358a5fffd913d9aa3605433",
      position: 1
    }
  }
}
\end{lstlisting}

\begin{lstlisting}[caption={A qua-compliant service output for mode \texttt{new}}, label={lst:quacompliantresult:new}]
{
  result : {
    unit : "metre",
    mode : "new"
  },
  callID : 213,
  duration : 3.0,
  scenario_id : 1337,
  timestamp_accessed : 1472137755,
  timestamp_modified : 1472137758
}
\end{lstlisting}

\clearpage
