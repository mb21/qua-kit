\section{QUA-Compliance}
\label{ch:quacompliance}

To make a service compliant for use in the QUA-Kit, a service is subject to additional constraints that are described here.

\subsection{QUA-View-Compliance}

\subsubsection{Registering}

\paragraph{Metadata}
When registering a service, the attribute \texttt{description} of the RemoteRegister service becomes obligatory. Furthermore, the boolean field \texttt{qua-view-compliant} must be provided and set to \texttt{True}.

Moreover, the optional field \texttt{constraints} is introduced and takes the following form:
\begin{lstlisting}
  "constraint" : {
      "some input" : ["foo", "bar"],
      "some other input" : {
          "min" : 0,
          "max" : 100,
          "integer" : True
      }
      // [...]
  }
\end{lstlisting}
For each input, a field in the constraints can be provided that specifies the domain of this input field further. Each constraint is either a list of values corresponding to an input field's type, or, for numeric values, a range. If no constraint is given for an input field, the input field's domain is considered unrestricted and every value of it's type is valid. There is a maximum of one constraint per input field allowed. If more than one constraint exists for a given input field, the service is considered broken and must be rejected.

The example above specifieds three inputs for a service. The first input argument is either \texttt{foo} or \texttt{bar}. The second one takes any whole number in the interval [0,100]. The third one can be any number since no constraint has been specified.

Note that a range constraint does not necessarily contain all three arguments min, max and integer.

\paragraph{Inputs}
While services originally can have arbitrary inputs, in QUA-compliant services, two input fields become obligatory: \texttt{mode} and \texttt{scenario\_id}. The mode is how the service should operate, normally it operates on a \texttt{scenario}, \texttt{objects} or \texttt{points} level; or creates a new scenario in the mode \texttt{new}.

The scenario-id specifies the scenario which should be analyzed by the service. If the mode is either objects or points, a binary attachment must be included in the service call:
\begin{itemize}
  \item \textbf{Points} are serialized to a binary file of \texttt{float32} values wherein three consecutive values correspond to one three dimensional vector.
  \item \textbf{Objects} are serialized to a binary file \texttt{ulong64} values wherein each value corresponds to one object id contained in the specified scenario.
\end{itemize}

\paragraph{Outputs}
The output of a service depends on the mode it operates in:

\begin{itemize}
  \item \texttt{scenario}:
  \begin{lstlisting}
{
  unit : string
  value : number
}
  \end{lstlisting}
  \item \texttt{objects}
  \begin{lstlisting}
{
  unit : string
  object_ids: [long]
  values : [number]
}
  \end{lstlisting}
  \item \texttt{points}
  \begin{lstlisting}
{
  unit : string
  values : [number]
}
  \end{lstlisting}
  \item \texttt{new}
  \begin{lstlisting}
{
  scenario_id : long // the id of the newly created scenario
  timestamp : long
}
  \end{lstlisting}
\end{itemize}

\clearpage
